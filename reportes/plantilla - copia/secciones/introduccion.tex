\section{Introducción}
Los sensores son dispositivos fundamentales en una amplia variedad de aplicaciones tecnológicas, ya que permiten la medición y detección de magnitudes físicas para su posterior procesamiento y análisis. Su desarrollo ha impulsado avances significativos en múltiples disciplinas, desde la automatización industrial hasta la instrumentación científica y los sistemas de seguridad.

En esta investigación se presentan los principales tipos de sensores, clasificándolos según su función y modo de operación. Se abordan sensores internos, como los de posición, velocidad, aceleración y fuerza, que permiten medir variables relacionadas con el movimiento y la interacción mecánica. También se analizan sensores externos, tanto de contacto como sin contacto, esenciales para la detección del entorno y la interacción con distintos elementos.

Comprender el funcionamiento, ventajas y aplicaciones de los sensores es clave para su correcta selección e implementación en distintos sistemas tecnológicos. A lo largo de este trabajo se exploran sus características y principios operativos, destacando su importancia en el desarrollo de soluciones eficientes y precisas en diversos campos de la ingeniería.

\section{Sensores}
Los sensores en los robots funcionan como los sentidos humanos, permitiéndoles recopilar información sobre su entorno para operar eficazmente. Deben detectar si han recogido un objeto, evitar obstáculos y ajustar su velocidad. También identifican características como peso, fragilidad o temperatura de los objetos. Para mover su efector final con precisión y aplicar la fuerza adecuada, los sensores en las articulaciones trabajan en conjunto con el controlador del robot, ya sea un microprocesador, computadora o microcontrolador.