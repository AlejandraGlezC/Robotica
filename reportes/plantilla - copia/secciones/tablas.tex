\section{Sensores externos} \label{sec:sensores ecternos}
Los sensores externos se dividen en sensores de contacto y sensores sin contacto. 
\subsection{Sensores de contacto}
Son aquellos que necesitan tocar físicamente un objeto para detectar su presencia o medir una magnitud.
\subsection*{Interruptores de límite}
\begin{itemize}
	\item \textbf{¿Qué hacen?} Detectan la presencia o posición de un objeto cuando este activa un mecanismo mecánico.
	\item \textbf{Principio de funcionamiento:} Consisten en un brazo mecánico o palanca que, al ser presionado, acciona un interruptor eléctrico.
	\item \textbf{Aplicaciones:} Detección de posición en máquinas CNC y robots industriales.
	Protección en sistemas de seguridad (por ejemplo, cuando una puerta está abierta o cerrada).
	Sistemas de final de carrera en actuadores.
	\item \textbf{Ejemplo:} Interruptor de límite tipo microswitch, como los usados en impresoras 3D para el eje Z.
\end{itemize}
\subsection*{Interruptores neumáticos}
\begin{itemize}
	\item \textbf{¿Qué hacen?} Detectan la presión de aire o vacío en un sistema neumático.
	\item \textbf{Principio de funcionamiento:} Usan un diafragma o válvula que se activa con cambios de presión.
	\item \textbf{Aplicaciones:} Control en sistemas de automatización neumática.
	Seguridad en prensas neumáticas y sistemas de frenado de emergencia.
	Sistemas de detección de flujo de aire.
	\item \textbf{Ejemplo:} Interruptor neumático utilizado en líneas de producción automatizadas.
\end{itemize}
\subsection*{Sensores piezoeléctricos}
\begin{itemize}
	\item \textbf{¿Qué hacen?} Detectan presión, fuerza o vibraciones y las convierten en señales eléctricas.
	\item \textbf{Principio de funcionamiento:} Se basan en el efecto piezoeléctrico, donde ciertos materiales generan un voltaje al ser sometidos a presión mecánica.
	\item \textbf{Aplicaciones:} Medición de impacto en pruebas de materiales.
	Sensores de vibración en maquinaria industrial.
	Micrófonos y captadores de sonido.
	\item \textbf{Ejemplo:} Sensores piezoeléctricos usados en guitarras eléctricas para captar sonido.
\end{itemize}
\subsection*{Transductores de presión}
\begin{itemize}
	\item \textbf{¿Qué hacen?} Miden la presión de un fluido (líquido o gas) y la convierten en una señal eléctrica.
	\item \textbf{Principio de funcionamiento:} Usan galgas extensométricas o elementos piezoeléctricos para medir la deformación causada por la presión.
	\item \textbf{Aplicaciones:} Monitoreo de presión en sistemas hidráulicos y neumáticos.
	Control de presión en motores y sistemas de refrigeración.
	Aplicaciones médicas (como en esfigmomanómetros digitales).
	\item \textbf{Ejemplo:} Sensor de presión MPX5700 usado en sistemas de control de presión.
\end{itemize}
\subsection{Sensores sin contacto}
Detectan la presencia, distancia o características de un objeto sin tocarlo.
\subsection*{Sensores de proximidad}
\begin{itemize}
	\item \textbf{¿Qué hacen?} Detectan la presencia de un objeto cercano sin contacto físico.
	\item \textbf{Tipos y funcionamiento:}
	
	\texttt{Inductivos:} Detectan objetos metálicos mediante un campo electromagnético.
	
	\texttt{Capacitivos:} Detectan objetos metálicos y no metálicos mediante cambios en la capacitancia.
	
	\texttt{Ópticos:} Usan luz infrarroja o láser para detectar objetos.
	\item \textbf{Aplicaciones:} Detección de piezas en bandas transportadoras.
	Sistemas de seguridad en maquinaria.
	Sensores de aparcamiento en automóviles.
	\item \textbf{Ejemplo:} Sensor inductivo LJ12A3-4-Z/BX usado en impresoras 3D.
\end{itemize}
\subsection*{Sensores de efecto Hall}
\begin{itemize}
	\item \textbf{¿Qué hacen?} Detectan la presencia de campos magnéticos.
	\item \textbf{Principio de funcionamiento:} Se basan en el efecto Hall, que genera una diferencia de voltaje en un material conductor cuando es atravesado por un campo magnético.
	\item \textbf{Aplicaciones:} Sensores de velocidad en motores.
	Controles de proximidad en robótica.
	Medición de corriente en circuitos eléctricos.
	\item \textbf{Ejemplo:} Sensor de efecto Hall A3144 para detectar imanes.
\end{itemize}
\subsection*{Sensores de microondas}
\begin{itemize}
	\item \textbf{¿Qué hacen?} Detectan movimiento mediante la emisión y recepción de ondas electromagnéticas de alta frecuencia.
	\item \textbf{Principio de funcionamiento:} Utilizan el efecto Doppler: cuando un objeto se mueve, la frecuencia reflejada cambia, lo que permite detectar su presencia y velocidad.
	\item \textbf{Aplicaciones:} Sensores de movimiento en alarmas de seguridad.
	Detección de vehículos en semáforos inteligentes.
	Sensores de radar en autos autónomos.
	\item \textbf{Ejemplo:} Sensor de microondas RCWL-0516 usado en sistemas de iluminación automática.
\end{itemize}
\subsection*{Sensores ultasónicos}
\begin{itemize}
	\item \textbf{¿Qué hacen?} Capturan imágenes y procesan información visual.
	\item \textbf{Principio de funcionamiento:} Utilizan cámaras con algoritmos de procesamiento de imagen para detectar formas, colores y movimientos.
	\item \textbf{Aplicaciones:} Inspección de calidad en líneas de producción.
	Reconocimiento facial en seguridad.
	Navegación de robots autónomos.
	\item \textbf{Ejemplo:} Cámara Intel RealSense para visión 3D.
\end{itemize}
\subsection*{Sensores láser}
\begin{itemize}
	\item \textbf{¿Qué hacen?} Miden distancias con alta precisión mediante un haz de luz láser.
	\item \textbf{Principio de funcionamiento:} Utilizan el tiempo de vuelo (ToF) de un pulso láser para calcular la distancia.
	\item \textbf{Aplicaciones:} Mapeo 3D en drones y vehículos autónomos.
	Medición de distancias en topografía.
	Sensores de seguridad en máquinas industriales.
	\item \textbf{Ejemplo:} Sensor LiDAR TFmini usado en robots para navegación autónoma.
\end{itemize}
\subsection*{Sensores de visión}
\begin{itemize}
	\item \textbf{¿Qué hacen?} Detectan movimiento mediante la emisión y recepción de ondas electromagnéticas de alta frecuencia.
	\item \textbf{Principio de funcionamiento:} Utilizan el efecto Doppler: cuando un objeto se mueve, la frecuencia reflejada cambia, lo que permite detectar su presencia y velocidad.
	\item \textbf{Aplicaciones:} Sensores de movimiento en alarmas de seguridad.
	Detección de vehículos en semáforos inteligentes.
	Sensores de radar en autos autónomos.
	\item \textbf{Ejemplo:} Sensor de microondas RCWL-0516 usado en sistemas de iluminación automática.
\end{itemize}

