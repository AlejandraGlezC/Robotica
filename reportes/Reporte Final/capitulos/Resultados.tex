\chapter{Resultados} \label{chap:resultados}

También mostrarán capturas de pantalla de cómo se ve en Gazebo el robot y en RViz, ya sea que funcione o no, para el cambio de trayectoria. Describirán lo que ocurre y por qué (si es que saben).

En la primera gráfica se muestra la evolución de la posición en el espacio cartesiano del efector final del robot a lo largo del tiempo. Las curvas representan las componentes X, Y y Z de dicha posición. Esta gráfica permite analizar cómo se desplaza el efector final en el espacio conforme avanzan los segundos de simulación, y verificar que siga la trayectoria deseada. \autoref{fig:cinematicadiferencial}

La segunda gráfica representa la velocidad lineal del efector final en cada una de las tres direcciones espaciales. Las componentes Vx, Vy y Vz indican con qué rapidez cambia la posición en cada eje. Esta información es útil para analizar si el robot está realizando movimientos suaves o si existen cambios bruscos de velocidad. \autoref{fig:cinematicadiferencial}

En la tercera gráfica se observa la aceleración lineal en los ejes X, Y y Z. Esta métrica refleja cómo cambia la velocidad lineal en el tiempo. Picos en esta gráfica podrían indicar esfuerzos mecánicos altos o comportamientos no deseados que podrían comprometer la estabilidad o el control del robot. \autoref{fig:cinematicadiferencial}


\begin{figure}
	\centering
	\includegraphics[width=0.5
	\linewidth]{img/cinematicadiferencial}
	\caption{Gráficas Cinemática Diferencial}
	\label{fig:cinematicadiferencial}
\end{figure}

La cuarta gráfica muestra la orientación angular del efector final expresada en términos de los tres ángulos de Euler: phi, theta y psi. Estos ángulos indican cómo rota el efector alrededor de sus propios ejes. Es clave para tareas donde no solo importa la posición, sino también cómo está orientada una herramienta o pinza montada en el extremo. \autoref{fig:cinematicadiferencial}

La quinta gráfica representa la velocidad angular, es decir, qué tan rápido cambian los ángulos de Euler con el tiempo. Se muestran las derivadas respecto al tiempo de phi, theta y psi. Esta gráfica permite verificar si los movimientos rotacionales son suaves y controlados. \autoref{fig:cinematicadiferencial}

Finalmente, la sexta gráfica ilustra la aceleración angular, que es la variación de la velocidad angular. Al igual que con la aceleración lineal, esta información es importante para detectar comportamientos bruscos o posibles problemas de control que afecten la precisión y el rendimiento del robot. \autoref{fig:cinematicadiferencial}

\begin{figure}
	\centering
	\includegraphics[width=0.5\linewidth]{img/cinematicadirecta}
	\caption{Cinemática Directa}
	\label{fig:cinematicadirecta}
\end{figure}


\begin{figure}
	\centering
	\includegraphics[width=0.5\linewidth]{img/ROS1}
	\caption{RViz del Robot}
	\label{fig:ros1}
\end{figure}





