\section{Fuentes Ochoa, Aislinn Alicia}
La integración del simulador ROS en este proyecto representó un paso importante para acercar la teoría de la robótica al entorno real de aplicaciones industriales y académicas. A través de la visualización del movimiento del brazo robótico dentro de un entorno tridimensional simulado, fue posible observar cómo la trayectoria programada en MATLAB se traduce en un comportamiento visual realista y físicamente coherente. Cada una de las capturas analizadas mostró una fase distinta del desplazamiento del robot, desde su posición inicial hasta la orientación final de su efector, permitiendo validar paso a paso la precisión del algoritmo y la fidelidad de la simulación. Esta representación visual no solo confirmó que los cálculos de posición y orientación se ejecutaron correctamente, sino que también demostró que los resultados son compatibles con entornos de simulación modernos, como los que se utilizan en ROS. Esta capacidad de integración multipropósito es fundamental en la formación de profesionales de robótica, ya que les permite diseñar, probar y validar sus algoritmos sin necesidad de acceso constante a hardware físico. En definitiva, este trabajo representa una base sólida para implementar sistemas de control más complejos en el futuro, integrando simulaciones, visualización y análisis de rendimiento en un solo flujo de trabajo.