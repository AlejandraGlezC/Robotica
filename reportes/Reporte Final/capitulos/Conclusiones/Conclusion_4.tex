\section{Peña Encinas, Ana Lourdes}
Durante el proceso de implementación del código, se evidenció la importancia de estructurar correctamente cada bloque funcional, desde la definición de la cinemática hasta la generación de la animación. Especialmente, se identificó el papel central que desempeñan el Jacobiano y su derivada en la interpretación del comportamiento cinemático del robot en el espacio operacional. Estas matrices no solo permiten traducir los movimientos articulares a movimientos del efector, sino que también sirven como base para entender cómo pequeñas variaciones en las articulaciones impactan en la trayectoria del extremo del robot. El hecho de que este análisis se haya complementado con gráficas detalladas de posición, velocidad, aceleración y orientación permitió una evaluación mucho más completa del sistema. Además, se hizo evidente la utilidad de contar con una estructura modular en el código, en donde cada función realiza una tarea específica y puede reutilizarse o modificarse de forma sencilla. Este enfoque de programación no solo facilita la comprensión del algoritmo, sino que también permite su extensión hacia sistemas más complejos o con mayores grados de libertad.