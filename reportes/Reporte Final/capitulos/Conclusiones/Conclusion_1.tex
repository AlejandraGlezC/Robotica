\section{Ceballos Portillo, Patsy}
El desarrollo y análisis de la cinemática diferencial del robot manipulador permitieron comprender de manera profunda el comportamiento dinámico del sistema en función del tiempo y las variables articulares. A través del cálculo preciso del Jacobiano geométrico y su derivada temporal, se logró establecer una relación directa entre las velocidades articulares y las velocidades lineales y angulares del efector final. Este proceso es fundamental no solo para simular correctamente el movimiento del robot, sino también para diseñar estrategias de control más avanzadas, como el control de trayectoria o el seguimiento en tiempo real. La obtención y representación gráfica de la velocidad y aceleración del efector en los tres ejes cartesianos permitieron validar el correcto funcionamiento del algoritmo, identificar posibles inconsistencias y evaluar el comportamiento esperado de la trayectoria. Estas gráficas no solo son una herramienta de validación técnica, sino también un recurso visual que facilita la interpretación de los resultados por parte de estudiantes, docentes e ingenieros, fortaleciendo así el puente entre la teoría matemática de la robótica y su aplicación práctica en simulaciones computacionales.