\section{Gonzalez Cueto, Alejandra Abigail}
La simulación de la cinemática directa del robot permitió visualizar de forma clara y dinámica el movimiento del manipulador en el espacio tridimensional, utilizando como base los valores articulares calculados previamente. Esta simulación resultó esencial para verificar si las transformaciones homogéneas implementadas permiten obtener correctamente la posición y orientación del efector final a lo largo del tiempo. A través del entorno de MATLAB, se logró animar el movimiento del robot con una frecuencia visualmente adecuada, lo cual ofreció una representación continua y fluida de su desplazamiento. Además, la posibilidad de generar y almacenar esta animación como un archivo de video aporta un recurso valioso para la documentación, el análisis posterior y la presentación del trabajo a terceros. Este tipo de visualizaciones permiten anticipar posibles problemas físicos, como colisiones, limitaciones en el rango de movimiento o comportamientos no deseados, sin la necesidad de contar con un prototipo físico. En conclusión, esta herramienta refuerza el valor de la simulación como un componente clave en el diseño, desarrollo y validación de sistemas robóticos.