\chapter{Marco Teórico} 
\label{chap:marco_teorico}

La robótica es una rama de la ingeniería que estudia el diseño, construcción, operación y aplicación de robots, así como los sistemas computacionales que los controlan. Un robot es una máquina reprogramable capaz de realizar una serie de acciones automáticamente o por control externo, con cierto grado de autonomía. El estudio de los robots implica diversas áreas del conocimiento, entre ellas la mecánica, la electrónica, la programación y la teoría del control.

La cinemática se encarga del estudio del movimiento de los cuerpos sin considerar las fuerzas que los causan. En robótica, se distingue entre:

\textbf{Cinemática directa:} determina la posición y orientación del efector final en función de los valores articulares.

\textbf{Cinemática inversa:} calcula los valores articulares necesarios para alcanzar una posición deseada del efector.


El análisis cinemático es esencial para programar tareas de posicionamiento precisas en robots manipuladores.

La dinámica estudia el comportamiento del robot considerando las fuerzas involucradas en su movimiento. Permite calcular los torques o fuerzas necesarias para lograr un cierto movimiento bajo condiciones dinámicas. Este análisis es fundamental para el diseño del sistema de control y para garantizar que el robot pueda ejecutar tareas de manera estable y segura.
Las propiedades dinámicas de un robot incluyen:

\textbf{Inercia:} resistencia de las masas del robot al cambio de movimiento.

\textbf{Centrífugas y coriolis:} fuerzas que aparecen en sistemas en movimiento.

\textbf{Par gravitacional:} el efecto del peso del robot sobre sus articulaciones.



