\section{Proceso de Cinemática} \label{sec:proceso_cinematica}

Aquí explicarán su código. Si quieren mostrar una parte, pueden hacerlo de la siguiente forma

% Si no quieres ponerle título al código, puedes dejarlo en blanco.
\begin{matlabcode}{matlab}
	function [q_sol, p_sol] = cinematica_inv(r, p_des, tol, max_iter, alpha, numMuestras)
\end{matlabcode}

Pero solo háganlo en partes muy específicas (las que van a explicar en ese momento). No copien todo el código ya que eso está en GitHub.

Si les sale el error \texttt{latexminted no se reconoce como un comando interno o externo, programa o archivo por lotes ejecutable}, deben tener instalado python y usar el siguiente comando.
\begin{terminal}{bash: Instalar minted en python con pip}
	pip install latexminted==0.5.1
\end{terminal}

\subsection{Cinemática Directa}
Explicar las partes importantes del código de la cinemática directa.
Para ver los resultados, ir al \autoref{chap:resultados}: Resultados, o determinada figura.
\subsection{Cinemática Diferencial}
Explicar las partes importantes del código de la cinemática diferencial.
Para ver los resultados, ir al \autoref{chap:resultados}: Resultados, o determinada figura.

En el codigo que se presenta, primero se inicializan varias variables para guardar los valores de posicion, orientacion, velocidad lineal y angular, y sus respectivas aceleraciones. Luego, se realiza un ciclo que recorre cada instante de tiempo. En cada paso, se actualiza la posicion del robot segun los valores articulares en ese momento y se calcula la posicion y orientacion del efector final.

Una de las lineas mas importantes del codigo es:
\begin{matlabcode}{matlab}
	vel_linear(:,:,k) = Jv(:,:,k) * dq(:,k);
\end{matlabcode}

Esta linea es clave porque calcula la velocidad lineal del efector final a partir del Jacobiano y de la velocidad articular. Es fundamental para entender como el movimiento de las articulaciones se traduce en un movimiento real del robot en el espacio.

Despues de esto, el programa tambien calcula la aceleracion utilizando la derivada del Jacobiano y las aceleraciones articulares. Los resultados se grafican, mostrando como varian la posicion, la velocidad y la aceleracion en los ejes X, Y y Z, asi como en los angulos de orientacion del robot. Estas graficas permiten analizar el comportamiento dinamico del robot y verificar si esta funcionando correctamente segun el modelo programado.

\subsection{Cinemática Inversa}
Explicar las partes importantes del código de la cinemática inversa.
Para ver los resultados, ir al \autoref{chap:resultados}: Resultados, o determinada figura.