\section{Proceso de Cinemática} \label{sec:proceso_cinematica}

En este capítulo se aborda el estudio de la cinemática de un robot manipulador, incluyendo la cinemática directa, la cinemática inversa y la cinemática diferencial. La cinemática directa permite calcular la posición y orientación del efector final a partir de los valores articulares del robot, mientras que la cinemática inversa busca determinar los ángulos articulares necesarios para alcanzar una posición deseada. Por otro lado, la cinemática diferencial se enfoca en el análisis de las velocidades y aceleraciones, utilizando el Jacobiano para relacionar el espacio articular con el espacio cartesiano. A lo largo del capítulo se explican detalladamente los códigos implementados para cada tipo de cinemática, acompañados de gráficas, simulaciones y visualizaciones que permiten validar y entender el comportamiento del sistema robótico.


\subsection{Cinemática Directa}

En esta sección del código se realiza la animación de la cinemática directa del robot. Primero, se cierra cualquier figura abierta con el comando close all y se define la cantidad de cuadros por segundo que tendrá la animación, en este caso treinta. Luego, se genera un vector de tiempo para la animación y se obtiene la trayectoria del robot aplicando la función de trayectoria, que devuelve los valores articulares que toma el robot en cada instante. 

Posteriormente, se crea la estructura grafica del robot, es decir, se define su representación visual en el entorno grafico de Matlab. Después, se genera un nombre automático para el archivo de video que incluirá la animación, utilizando la fecha y hora actuales, lo cual permite guardar cada ejecución sin sobrescribir archivos anteriores. Aunque se incluyen líneas para generar un video de la animación, estas se encuentran comentadas, lo cual significa que no se ejecutan a menos que el usuario lo decida. 

La animación se genera mediante un ciclo que recorre cada instante de tiempo y actualiza la configuración del robot usando los valores articulares correspondientes, y luego dibuja el robot en su nueva posición. Si se desea guardar el video, también se capturaría el fotograma actual y se escribiría en el archivo de video. Esta parte del código permite visualizar el movimiento del robot a lo largo de la trayectoria calculada, facilitando el análisis del comportamiento de su cinemática directa. Para visualizar el resultado vea la \autoref{fig:cinematicadirecta}.

\begin{matlabcode}{Cinemática Directa}
	for k = 1:length(t_animacion);
\end{matlabcode}


\subsection{Cinemática Diferencial}

En el codigo que se presenta, primero se inicializan varias variables para guardar los valores de posicion, orientacion, velocidad lineal y angular, y sus respectivas aceleraciones. Luego, se realiza un ciclo que recorre cada instante de tiempo. En cada paso, se actualiza la posicion del robot segun los valores articulares en ese momento y se calcula la posicion y orientacion del efector final.

Una de las lineas mas importantes del codigo es:
\begin{matlabcode}{Cinemática Diferencial}
	vel_linear(:,:,k) = Jv(:,:,k) * dq(:,k);
\end{matlabcode}

Esta linea es clave porque calcula la velocidad lineal del efector final a partir del Jacobiano y de la velocidad articular. Es fundamental para entender como el movimiento de las articulaciones se traduce en un movimiento real del robot en el espacio.

Despues de esto, el programa tambien calcula la aceleracion utilizando la derivada del Jacobiano y las aceleraciones articulares. Los resultados se grafican, mostrando como varian la posicion, la velocidad y la aceleracion en los ejes X, Y y Z, asi como en los angulos de orientacion del robot. Estas graficas permiten analizar el comportamiento dinamico del robot y verificar si esta funcionando correctamente segun el modelo programado. Las gráficas obtenidas se pueden ver visualizadas en la  \autoref{fig:cinematicadiferencial}.

\subsection{Cinemática Inversa}
Explicar las partes importantes del código de la cinemática inversa.
Para ver los resultados, ir al \autoref{chap:resultados}: Resultados, o determinada figura.