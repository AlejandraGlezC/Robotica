\section{Control} \label{sec:control}
En cuanto al control del robot, aunque no se desarrolló un apartado completamente detallado durante este semestre, es importante entender algunos principios generales sobre cómo funciona esta parte esencial. El control de un robot industrial, como el ABB IRB1200-7/70, consiste en regular los movimientos de sus articulaciones (joints) para que pueda realizar tareas específicas, como mover su brazo, abrir o cerrar las pinzas, o seguir una trayectoria definida.

En términos simples, el robot recibe un conjunto de entradas (inputs), que pueden ser señales de posición, velocidad o esfuerzo, y a partir de ellas genera las salidas necesarias (outputs) para que los motores actúen de la forma adecuada. Por ejemplo, si queremos que el brazo se mueva a un punto específico en el espacio, el sistema de control debe calcular cuánto tiene que rotar cada articulación y enviar la señal precisa a cada motor.

Este proceso involucra varios elementos de control, como los controladores PID (Proporcional, Integral, Derivativo), que son ampliamente usados para ajustar las señales que llegan a los motores y garantizar que el robot se mueva de forma suave, precisa y sin errores. En el caso de las simulaciones con ROS, se pueden usar controladores predefinidos configurados en los archivos de ROS Control, los cuales permiten asignar comandos de posición, velocidad o esfuerzo a cada joint del modelo URDF.

Además, para coordinar movimientos más complejos, como sincronizar las dos pinzas para que se cierren al mismo tiempo, se utilizan sistemas de planificación de movimientos como MoveIt, que calculan las trayectorias y envían los comandos de control adecuados para que el robot las siga. En este proyecto, aunque no se profundizó del todo en el diseño y ajuste de los controladores, sí se configuraron los inputs básicos necesarios para ejecutar movimientos y verificar que las simulaciones respondieran correctamente a las órdenes dadas desde ROS.

En resumen, el control del robot es el puente entre lo que queremos que haga (el comando o input) y lo que realmente hace (el movimiento o output). Aunque esta área tiene mucha profundidad y aspectos avanzados como la compensación de dinámica, la retroalimentación sensorial y la optimización de trayectorias, en este proyecto se abordaron las bases necesarias para lograr movimientos controlados y funcionales dentro del entorno de simulación.
