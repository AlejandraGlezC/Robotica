\section{Simulación} \label{sec:simulacion}
En este proyecto, se llevó a cabo una simulación completa del robot ABB IRB1200-7/70, comenzando desde la obtención del modelo CAD hasta su integración y ejecución en ROS. A continuación, se describen los pasos realizados para lograr una simulación funcional y detallada.

Primero, se descargó el modelo CAD del robot desde la plataforma GrabCAD, la cual ofrece una amplia biblioteca de diseños mecánicos. El modelo seleccionado fue el ABB IRB1200-7/70, uno de los robots industriales más comunes para tareas de manipulación. Este modelo incluía todos los archivos necesarios en formato de SolidWorks, lo que permitió visualizar y manipular cada uno de los componentes mecánicos, como motores, eslabones y pinzas, dentro de un entorno CAD.

El siguiente paso fue la conversión del modelo de SolidWorks al formato URDF (Unified Robot Description Format), que es el estándar en ROS para describir la estructura, geometría, cinemática y propiedades físicas de los robots. Para realizar esta conversión, se siguieron las instrucciones proporcionadas en el repositorio de GitHub, el cual contiene la guía del proceso usando la herramienta sw2urdf. Esta herramienta es un complemento que se integra dentro de SolidWorks y permite exportar ensamblajes completos al formato URDF, generando automáticamente los archivos necesarios para ROS, incluyendo las mallas visuales (formato STL) y las propiedades físicas (como masas e inercias).


Durante el proceso de exportación, fue importante asegurarse de que cada pieza estuviera correctamente definida: se revisaron las conexiones entre los eslabones, se verificó que las masas y los centros de gravedad estuvieran correctamente asignados, y se aseguraron los materiales adecuados para una simulación física realista. Este paso fue fundamental para que, al momento de correr la simulación, los movimientos del robot fueran precisos y coherentes con los cálculos cinemáticos y dinámicos.

En cuanto al entorno de simulación, se utilizó el sistema operativo Ubuntu, que es la plataforma principal para trabajar con ROS. Este sistema se ejecutó ya sea de manera nativa en una computadora o mediante soluciones como máquinas virtuales (por ejemplo, VirtualBox) o Windows Subsystem for Linux (WSL) en caso de trabajar desde un entorno Windows. La instalación de ROS en Ubuntu permitió acceder a herramientas clave para la simulación y el control del robot.

Una vez integrado el modelo URDF en ROS, se procedió a simular el movimiento de las dos pinzas del robot. Esto requirió configurar correctamente los controladores en los archivos de configuración, ajustando los parámetros de control de posición, velocidad o esfuerzo, según el tipo de movimiento deseado. Se trabajó para sincronizar los movimientos de ambas pinzas, de modo que pudieran moverse al mismo tiempo en la simulación. Además, se exploró la posibilidad de implementar un electroimán como herramienta de agarre, considerando tanto su representación visual como su comportamiento físico, aunque su integración final dependió de los alcances y recursos del proyecto.

La simulación se ejecutó utilizando diversas herramientas integradas en ROS. RViz fue la herramienta principal para la visualización del robot, permitiendo verificar la correcta articulación y orientación de cada parte del modelo, así como observar en tiempo real los movimientos planificados. Gazebo proporcionó el entorno físico de simulación, ofreciendo un espacio tridimensional donde el robot podía interactuar con objetos y superficies, permitiendo validar aspectos dinámicos como fuerzas, colisiones y fricción. Finalmente, MoveIt se empleó para la planificación de movimientos y trayectorias, facilitando la programación de tareas complejas y la ejecución coordinada de los movimientos del brazo y las pinzas. La imagen de la simulación se puede visualizar en la  \autoref{fig:ros1}

En resumen, todo este proceso permitió realizar una simulación detallada y funcional del robot ABB IRB1200-7/70, integrando aspectos mecánicos, cinemáticos y dinámicos, y utilizando herramientas avanzadas del ecosistema ROS. Esta simulación fue clave para validar los algoritmos de control desarrollados en el proyecto, así como para preparar al equipo para futuras implementaciones en un entorno físico real.