\chapter{Introducción} \label{chap:introduccion}

La robótica es una disciplina multidisciplinaria que ha revolucionado la manera en que interactuamos con la tecnología, fusionando conocimientos de mecánica, electrónica, programación y control para el diseño, análisis y operación de sistemas automatizados. En el contexto de la ingeniería moderna, los robots no solo representan herramientas sofisticadas, sino también soluciones prácticas a problemas complejos en sectores como la manufactura, la medicina, la exploración espacial, la agricultura, entre muchos otros.

Este reporte tiene como propósito documentar el análisis completo realizado sobre un robot seleccionado durante el desarrollo de la materia de Robótica. A través de esta práctica, se abordaron temas fundamentales para la comprensión integral de los sistemas robóticos, desde la caracterización estructural y dinámica del robot, hasta el desarrollo de su cinemática y control. El enfoque metodológico aplicado en este trabajo permite reforzar conceptos clave como la formulación de parámetros Denavit-Hartenberg, el análisis de las articulaciones, la identificación de motores y materiales, así como la evaluación de las limitaciones físicas del sistema.

Durante el desarrollo del proyecto, se realizó una caracterización detallada del robot, comenzando con la recolección de datos esenciales sobre cada una de sus articulaciones, incluyendo los límites de posición, velocidad, aceleración, fuerzas y coeficientes de fricción. A partir de esta información, se elaboró una tabla DH y se describió la geometría del robot, lo cual permitió construir un modelo matemático preciso que sirvió como base para el análisis cinemático. Asimismo, se identificaron y describieron las partes principales del robot, tales como motores y eslabones, especificando las propiedades relevantes como torque, velocidad máxima y el tipo de material utilizado en su fabricación.

En la etapa de cinemática, se desarrollaron los modelos directo e inverso, fundamentales para determinar la posición y orientación del efector final con base en los parámetros articulares, o viceversa. Posteriormente, se implementó un sistema de control para gobernar el movimiento del robot de manera eficiente, utilizando un diagrama a bloques que ilustra el flujo de información y la interacción entre los distintos componentes del sistema. El control es esencial no sólo para garantizar la precisión del movimiento, sino también para asegurar la estabilidad y seguridad del robot durante su operación.

Finalmente, en la sección de resultados se presentan las gráficas obtenidas a partir del análisis cinemático y dinámico, incluyendo las trayectorias deseadas y actuales del efector final en el espacio cartesiano, así como la evolución temporal de los parámetros dinámicos. Estas gráficas permiten evaluar el desempeño del sistema y detectar posibles áreas de mejora en el diseño o la implementación del control.

A través de este proyecto, no solo se reforzaron habilidades técnicas, sino que también se fomenta el trabajo colaborativo, el pensamiento crítico y la capacidad de aplicar teoría en contextos prácticos. El conocimiento adquirido representa una base sólida para enfrentar futuros retos en el ámbito de la automatización y el desarrollo de soluciones robóticas innovadoras.
